\chapter{Conclusion and outlook}
\section{Conclusion}
%A new layout scheme of the solar dish system by using SEA were proposed. Connection type of the engines may change the flow rates and temperatures of the fluids, as a result the performance of the SEA will be different depending on the connection schemes. In order to compare performance of SEAs with different arrangements, five basic connection types of SEA were summed up according to flow type and flow order. 
%
%Analytical Stirling engine model was created to develop the SEA models for the investigation of influence of connection types. Imperfect regeneration and cycle irreversibility of Stirling engine cycle and heat exchange process between fluids and engine were considered in the model. Algorithm to numerically solve different connection types of SEA was developed. The model was evaluated by considering the prototype GPU-3 Stirling engine as a case study. Result shows that the proposed model predicted the performance with higher accuracy than Simple model~\cite{Urieli1984} and Simple II model~\cite{Strauss2010}. 
%
%Models of SEAs were developed to calculated the performance under different parameters to find out the impacts of $T_{i,h}$, $\dot{m}_hc_{p,h}$, $\dot{m}_cc_{p,c}$ and $n_{se}$ on different connection types. It was found that, as expected, decrease $T_{i,h}$ and $q_{m}c_{p}$ will weaken the performance of SEA of all connection types. However, for some connection types, there exists a critical temperature below which some engines stop working. This needs to be considered for SEA connection type selection, especially when $T_{i,h}$ is low. For given heating and cooling fluids, Type 2 has the best performance and adaptability. Type 2 and Type 3 have similar performance under different parameters ($T_{i,h}$, $T_{i,c}$ and $\dot{m}c_p$), which means the flow order has little influence on the performance of an SEA. SEA of serial flows (Type 3) has the best performance and adaptability under different parameters. Given heating and cooling fluids, using serial flow is the best choice for the connection type of an SEA. %This means the new arrangement of dish-Stirling system in Figure~\ref{fig:Dish_SEA} may have better performance than the traditional arrangement, which can be considered as a particular case of Type 1.
%
%It is important to note that, in the future researches, the experiments of influence of connection type on SEA's performance can be carried out to verify the conclusions in this thesis.
This chapter is needed to conclude the overall goal of our research. Considering the advantages and disadvantages of the existing solar thermal power generation technologies, a novel idea of energy cascade collection and energy cascade utilization for solar thermal power generation is put forward. Different types of collectors and thermodynamic cycles were used in the cascade system. The research of the cascade system is carried out with the selection of the system topology, the construction of the system model, the optimization of the system model and parameters, and the comparison with the independent system. The main works are concluded as follow:
\begin{enumerate}[label=(\arabic*)]
  \item The topological structure of solar thermal cascade power generation system was proposed and analyzed. According to the analysis of thermal characteristics and the working characteristics of each component in the system, rationally arranged topological structures of cascade system were proposed. Specifically, by selecting and analyzing the basic system, the Rankine cycle fluid, the solar chimney, the cascade of the various types of collectors, the direct steam production system, the heat exchanger between the different working fluids, the heat between the different cycles Recovery, etc., to identify the representative of the project for the representative of the solar heat and heat cascade power generation system program. That is, on the basis of the slotted Rankine cycle and the disc-type Stirling cycle, the traditional dish-type Stirling machine is changed to hot air as a closed circulation system for the media, and the hot air heated by the dish collector The Stirling unit is used to heat the Stirling unit and use the condensed water of the Rankine cycle to cool the Stirling machine to utilize the hot air flowing from the Stirling unit to achieve overheating for the steam.
  \item Mechanism models were established for the components of solar thermal power generation system. The mechanism mathematical models were developed according to the operation mechanism of the target object and physical equations. The key components in the system, such as collectors, steam generating system, steam turbine and Stirling engine, were modeled with details. The mathematical model of each component is a model verified by the classical theory or a large number of experimental data, which is the basic of the model of the cascade solar thermal power generation system. Heat loss models were established for the receivers of trough collector and dish collector. For the Stirling engine, based on the reasonable simplification and hypothesis, the model of the Stirling machine considered various losses and irrevisibilities was developed. The component models were developed in MATLAB by using object-oriented method. It makes full use of inheritance and polymorphism to ensure both the independence and the relevance of the components.
  \item A solar thermal power generation system design software was designed and the solar thermal power generation system models were developed. Based on the selected solar thermal cascade generation systems, solar thermal cascade generation system models were established based on the model of each component in the systems. The object-oriented features of inheritance, combination and polymorphism were used for the model development. The change rules of the main parameters and the performance indexes under the coupling of external and internal factors were studied. The change mechanism was studied and the calculation method of its performance characteristics was established. After setting up the components, setting the parameters and compiling the environment, the thesis completes the system construction of each system scheme, and finally completes the simulation system of solar thermal cascade generation based on MATLAB with the copyright of independent computer software. 
  \item Simulation and optimization of cascade solar thermal power generation system model. Based on the study of the performance characteristics of solar thermal cascade generation system, the system is optimized and the structure is reconstructed. In particular, by analyzing the steam generation system of the system, a method of staged heating is proposed to reduce the heat transfer temperature difference in the steam generating system by changing the mass flow rate of the heat conduction oil, effectively reducing the heat generated during the heat exchange process in the steam generating system. Which can improve the efficiency of the whole system. Based on Stirling unit in cascade system, five kinds of basic arrangement forms of Stirling unit are summarized, and the difference of unit efficiency and output power under various arrangement forms is analyzed, and a given cold and heat source fluid Stirling unit under the conditions of the best arrangement.
\end{enumerate}

\section{Innovation}
%多种型式的集热器和循环并用
%
%斯特林机的排布
%
%MERS的应用

\begin{itemize}
  \item Usage of different types of collectors and different thermodynamic cycles is used in this research.
  \item Multi-stage exerge lose reduction system is applied to reduce the temperature difference between oil and water in the steam generating system.
  \item Influence of the arrangement of Stirling engine array in the cascade system is analyzed.
\end{itemize}

\section{Outlooks}
In this research, effective topologies of the proposed cascade system were designed, models of the systems developed based on the detailed component models, simulation of the cascade system and corresponding stand alone systems were carried out and the results were analyzed. However, there are many points valuable for further research.
\begin{itemize}
  \item Solar power tower technology is gaining more and more attention, which represents the future of CSP technology and needs to be investigated in the future work. 
  \item With the development of solar collectors, combination of different types of collectors will overcome the current drawbacks and deserves more focus.
  \item Economical analysis of the cascade system is required for the implement of the technology.
\end{itemize}