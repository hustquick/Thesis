\zhabstract{
随着化石能源的消耗和环境问题的凸显,太阳能作为一种新能源,具有分布广泛、总量巨大、取之不竭、无污染的特点,越来越受到世界各国的重视,被广泛认为是未来最有潜力替代传统化石能源的清洁能源。在发电领域,太阳能光热发电是除了太阳能光伏发电之外的另一种发电形式。与光伏发电相比,光热发电因具有发电平稳,电网兼容性友好,易于与现有化石燃料电厂组合等优点而受到越来越多的关注。已经商业应用的太阳能光热发电技术分为槽式集热发电、碟式集热发电和塔式集热发电三种。三种发电技术各有优缺点:槽式集热发电应用最广,成本较低,但效率也较低;碟式集热发电规模较小,多用于分布式发电;塔式集热发电规模较大,成本较高,目前处于快速发展阶段。综合利用现有发电技术的优缺点,在能量梯级收集和能量梯级利用的思想上,提出采用多种集热发电方式和多种热功循环的梯级系统,是实现大规模太阳能光热发电的一种新颖的可行的技术方案。

本课题以国家国际合作项目专项“太阳能梯级集热发电系统关键技术合作研究”为背景,目标是研究太阳能光热发电装置,利用各种传统型式的太阳能光热发电系统的优缺点以及热力特性,提出并组建、优化太阳能梯级集热发电系统,为探索出可大规模高效率利用太阳能的光热发电技术提供新的方案。主要研究内容和结论包括:

首先,提出太阳能光热梯级集热发电系统的拓扑结构。通过热力特性分析,结合系统中各部件的工作特点,合理布局太阳能光热梯级集热发电系统,利用不同热功循环实现不同品位的能量的梯级利用。合理的梯级发电系统方案才能充分利用发电系统中各部件的性能特点,为创建高效率的太阳能光热梯级发电系统提供基础。本文针对系统中的各组件,组建了多种可行的梯级集热系统拓扑结构。经过系统评估、参数选取、初步计算、方案比较,确定了两种具有代表性的太阳能光热梯级发电系统方案。一种方案同时选用水工质朗肯循环和斯特林循环,利用给水来冷却斯特林机冷腔,回收利用斯特林机放出的热量;另一种方案选用多级有机工质朗肯循环,利用上一级的凝集热来加热下一级的循环工质,实现能量的梯级利用。

其次,针对太阳能光热梯级集热发电系统的各部件建立机理模型。依据目标对象的运行机理,根据物理平衡方程,对系统中的各部件,尤其是系统中的关键部件,如集热器、蒸汽产生系统、汽轮机、斯特林机等,建立起数学模型。各部件的数学模型是经由经典理论或是大量实验数据验证的模型,是组建光热梯级集热发电系统模型的基础。对于槽式集热器的集热管和碟式集热器的集热器,建立了热损失模型;对于斯特林机,基于合理的简化和假设,推导出了考虑了各种热损失和不可逆因素的斯特林机模型。各部件模型使用MATLAB语言编写,采用面向对象的方法,充分利用了继承、多态等特性,保证了各部件之间既具有独立性又具有关联性。

再次,组建太阳能光热梯级集热发电系统模型。根据所选择的太阳能光热梯级发电系统方案,基于建立好的系统中各部件的模型,利用面向对象语言的继承、组合、多态等特点,组建起梯级集热发电系统模型。研究系统在外部及内部因素的耦合作用下主要参数及性能指标的变化规律,掌握其变化机理,建立其性能特性的计算方法。经过组建部件,设置参数,编译环境,完成了各系统方案的系统组建工作,最终完成了拥有自主计算机软件著作权的基于MATLAB的太阳能光热梯级发电的模拟系统。系统中各部件相对独立,便于更换或改进部件模型;各系统模型的计算结果可以以单个对象的方式方便地查看系统中各个部件的关键参数。

然后,模拟并优化太阳能光热梯级集热发电系统模型。在太阳能光热梯级发电系统性能特性研究的基础上,对系统进行流程优化、结构重构。具体地,通过对系统的蒸汽发生系统进行分析,提出了分阶段加热方法,通过改变导热油的质量流量降低蒸汽发生系统中的传热温差,有效降低了蒸汽发生系统中换热过程中产生的㶲损,进而可以提高整个系统的效率。针对梯级系统中的斯特林机组,总结了斯特林机组所具有的五种基本排列形式,并分析了各种排列形式下机组的效率和输出功率的差异,得到了给定冷热源流体条件下斯特林机组最佳的排列方式。

最后,优化太阳能光热梯级集热发电系统的运行参数。针对特定结构方案和运行模式,以梯级发电系统的性能参数和经济指标为目标函数,选择合理的可调节参数,确立各种约束条件,利用现代优化方法,如基因算法、蚁群算法,完成系统的参数优化分析,以及对于独立系统的对比分析。分析结果表明,太阳能光热梯级集热发电系统在一定的参数条件下,相比其对应的独立系统,具有更高的总体光电转换效率。在太阳直射强度为$700\,\mathrm{W/m^2}$,碟式集热器出口空气温度为$800\mathrm{^\circ C}$的条件下,方案1所选用的太阳能光热梯级集热发电系统比对应的独立系统效率提升$5.2\%$,方案2所选用的太阳能光热梯级集热发电系统比对应的独立系统效率提升$15.3\%$。

}
\zhkeywords{槽式集热器,碟式集热器,朗肯循环,斯特林循环,斯特林机组,梯级发电}

\enabstract
{
With the increasing awareness of the problem of fossil energy consumption and environmental pollution, solar energy as a renewable energy, which has the advantages of widely distribution, huge amount, inexhaustible and no pollution, has received much attention by many countries and been regarded as the best potential candidate of fossil energy. Concentrating solar thermal power generation is another form of power generation technology except solar photovoltaic power generation. Compared to solar photovoltaic, solar thermal power is gaining more attention for its advantages as smooth power generation, good grid compatibility, easy to integrate with existing fossil power plant.

Commercial solar thermal power generation technology is divided into trough collector power generation, dish collector power generation and solar tower power generation. These three types of power generation technologies have their own advantages and disadvantages: trough collector power generation is the most widely used one, its cost is low, however its efficiency is also low; dish collector power generation has high efficiency and smaller capacity, it is used for distributed generation widely; solar power tower, with large scale, high efficiency and high cost, is currently in rapid development stage. Based on the idea of ​​energy cascade collection and energy cascade utilization, this thesis proposed a cascade system that uses different collector power generation technologies and different thermodynamic cycles, which may be a new and feasible technology to realize large-scale solar thermal power generation.

The research is based on the national cooperation project "Collaborative research on key technologies to produce electricity by cascade utilization solar thermal energy" as the background. The objective of this project is to research the equipment of solar thermal power generation system, to propose, develop and optimize a solar thermal cascade system depending on the advantages and disadvantages of the solar thermal power generation technologies, and to explore a new feasible technology for large-scale solar thermal power generation. The main contents and conclusions of this thesis are as follows:

Firstly, the topological structures of solar thermal cascade power generation system were proposed. According to the analysis of thermal characteristics and the working characteristics of each component in the system, rationally arranged topological structures of cascade system were proposed. These systems use different thermodynamic cycles to utilize energy in different temperature zones. A reasonable cascade generation system can make full use of the performance characteristics of the components in the power generation system and provide the foundation for higher efficiency solar thermal cascade generation systems. In this thesis, several schemes of feasible topological structures of solar thermal cascade system were set up according to the components in the system. After system evaluation, parameter selection, preliminary calculation and scheme comparison, two representative typical schemes were determined. In one scheme, both Rankine cycle (water as the working fluid) and Stirling cycle are used for power generation. Cooling water of the Rankine cycle is used to cool the hot end of the Stirling engines to recover the released heat. In the other scheme, multiple organic Rankine cycles are used for power generation. Condensation heat of upper cycle is absorbed by lower cycle for energy cascade utilization.


Secondly, mechanism models were established for the components of solar thermal power generation system. The mechanism mathematical models were developed according to the operation mechanism of the target object and physical equations. The key components in the system, such as collectors, steam generating system, steam turbine and Stirling engine, were modeled with details. The mathematical model of each component is a model verified by the classical theory or a large number of experimental data, which is the basic of the model of the cascade solar thermal power generation system. Heat loss models were established for the receivers of trough collector and dish collector. For the Stirling engine, based on the reasonable simplification and hypothesis, the model of the Stirling machine considered various losses and irrevisibilities was developed. The component models were developed in MATLAB by using object-oriented method. It makes full use of inheritance and polymorphism to ensure both the independence and the relevance of the components.

Thirdly, the solar thermal cascade generation system models were developed. Based on the selected solar thermal cascade generation systems, solar thermal cascade generation system models were established based on the model of each component in the systems. The object-oriented features of inheritance, combination and polymorphism were used for the model development. The change rules of the main parameters and the performance indexes under the coupling of external and internal factors were studied. The change mechanism was studied and the calculation method of its performance characteristics was established. After setting up the components, setting the parameters and compiling the environment, the thesis completes the system construction of each system scheme, and finally completes the simulation system of solar thermal cascade generation based on MATLAB with the copyright of independent computer software. The system components are relatively independent, easy to replace or improve the parts model; the results of the calculation of the system model can be a single object to easily view the various components of the system key parameters.

Then, simulation and optimization of cascade solar thermal power generation system model. Based on the study of the performance characteristics of solar thermal cascade generation system, the system is optimized and the structure is reconstructed. In particular, by analyzing the steam generation system of the system, a method of staged heating is proposed to reduce the heat transfer temperature difference in the steam generating system by changing the mass flow rate of the heat conduction oil, effectively reducing the heat generated during the heat exchange process in the steam generating system. Which can improve the efficiency of the whole system. Based on Stirling unit in cascade system, five kinds of basic arrangement forms of Stirling unit are summarized, and the difference of unit efficiency and output power under various arrangement forms is analyzed, and a given cold and heat source fluid Stirling unit under the conditions of the best arrangement.

Finally, the operating parameters of solar thermal cascade power generation system are optimized. According to the specific structural scheme and operation mode, the performance parameters and economic indexes of the cascade generation system are taken as the objective function, reasonable adjustable parameters are selected, various constraints are established, and modern optimization methods such as genetic algorithm and ant colony algorithm are used to complete the system Parameter optimization analysis, as well as the independent system for comparative analysis. The results show that solar thermal cascade power generation system has higher overall photoelectric conversion efficiency under certain parameter conditions than its corresponding independent system. Under the condition of direct solar radiation intensity of 700\,$\mathrm{W/m^2}$ and dish type collector outlet air temperature of 800$\mathrm{^\circ C}$, the solar thermal cascade power generation system of Scheme 1 is better than the corresponding The efficiency of stand-alone system is increased by 5.2\%. The solar thermal cascade power generation system selected in Scheme 2 is 15.3\% more efficient than the corresponding independent system.
}
\enkeywords
{parabolic trough collector, parabolic dish collector, Rankine cycle, Stirling cycle, Stirling engine array, cascade powering}