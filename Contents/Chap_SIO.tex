\chapter{Summary and outlook}
\section{Summary}

This chapter is needed to conclude the overall goals of our research. Considering the advantages and disadvantages of the existing solar thermal power generation technologies, a novel idea of energy cascade collection and energy cascade utilization for solar thermal power generation is put forward. Different types of collectors and thermodynamic cycles were used in the cascade system. The research of the cascade system is carried out with the selection of the system topology, the construction of the system model, the optimization of the system model and parameters, and the comparison with the independent system. The main works are concluded as follow:

\begin{enumerate}[label=(\arabic*)]
  \item Topological structures of cascade solar thermal power generation systems were proposed. According to the analysis of thermodynamic characteristics and the operating behavior of each component in the system, reasonable arranged topological structures of cascade systems were proposed. These systems use different thermodynamic cycles to harness the energy of different temperature zones. A reasonable cascade generation system can make full use of the mechanism models of power generation system to provide a basis for a more efficient cascade solar thermal generation system. In this thesis, several schemes of feasible topological structures of solar thermal cascade system were set up according to the mechanism model of each component. After system evaluation, parameter selection, preliminary calculation and scheme comparison, two representative typical schemes were determined. In one scheme, both steam Rankine cycle and Stirling cycle are used for power generation. Condensation water of the Rankine cycle is used to cool the hot end of the Stirling engines to recover the released heat. In the other scheme, multiple organic Rankine cycles are used for power generation. Condensation heat of upper cycle is absorbed by lower cycle for energy cascade utilization.
  \item Mechanism models were established for the components of solar thermal power generation system. The mechanism mathematical models were developed according to the physical equations and operation features of the target object. The key components in the system, such as collectors, steam generating system, steam turbine and Stirling engine, have been analyzed for detailed modeling. The mathematical model of each component is a model verified by classical theory or a large number of experimental data, which is the basic of a cascade solar thermal power generation system model. Heat loss models were established for the receivers of trough collector and dish collector. For Stirling engine, based on reasonable simplification and hypothesis, a model of the Stirling machine considered various losses and irrevisibilities was developed. The component models using object-oriented method were developed in MATLAB. It makes full use of inheritance and polymorphism to ensure both independence and relevance of the components.
  \item A solar thermal power generation system design software was designed and the cascade solar thermal generation system models were developed. System models of the selected cascade solar thermal thermal generation systems were established based on the model of each component in the systems. The object-oriented features of inheritance, combination and polymorphism were used for the model development. The variation rules and performance indexes of main parameters under the coupling of external and internal factors were studied. The change mechanism was studied and the calculation method of its performance characteristics was established. After component layout, parameter setting and environment selection, the thesis completed the system development of each system scheme, and finally developed the simulation system of cascade solar thermal generation based on MATLAB with the copyright of independent computer software. The system components are relatively independent, easy to replace or improve the component model; calculation results of the system model exist in all objects of the system, so that the key parameters of each component can be clearly and conveniently viewed. 
  \item Simulation and optimization of cascade solar thermal power generation system model were carried out. Based on the research of performance characteristics of cascade solar thermal power generation system, the system is optimized and the structure is rebuilt. In particular, by analyzing the steam generating system of the system, a staged heating method is proposed, which can reduce the temperature difference in the steam generating system during steam generation by changing the mass flow rate of the heat transfer oil and  effectively reduce exergy loss during the process. It helps to improve the efficiency of the whole system. 
Considering the features of the Stirling engines in the cascade system, five basic arrangements of Stirling engine array are summarized, and the differences of Stirling engine array efficiency and output power under various layouts were analyzed. The best arrangement of Stirling units was given under the condition of given fluid of cold and heat sources.
	\item Operating parameters of cascade solar thermal power generation system were optimized. 
According to the specific structure of the program and operation mode, the appropriate stand-alone systems for comparative analysis was selected for performance comparison. Analysis of the influence of various parameters on the efficiency difference between cascade system and its corresponding stand-alone systems was conducted. The results show that cascade solar thermal power generation system has higher overall solar-to-electric conversion efficiency under certain parameter conditions than its corresponding independent system. Under the condition of direct normal irradiance of 700\,$\mathrm{W/m^2}$ and dish collector outlet air temperature of 800$\mathrm{^\circ C}$, the proposed cascade solar thermal power generation system is 5.2\% more efficient than its corresponding stand-alone system.
	\item A solar thermal power generation test platform was built, and the relevant experimental work was carried out. Special experiment cases considering the features of solar irradiance were designed to investigate the impact of different factors on the system performance. The influences of solar irradiance, flow rate and inlet temperature of the working fluid on the performance of the collectors were investigated. The analysis of experimental data and simulation results shows that, under the relevant test conditions, the thermal efficiency of trough collectors is between 58\% and 64\%, and that of trough collectors is between 63\% and 68\%. The experiment also validated the established trough collector and dish collector models.
\end{enumerate}

\newpage
\section{Innovation}

\begin{itemize}
  \item Usage of different types of collectors and different thermodynamic cycles in one cascade system is proposed in this research. In this way, the working characteristics of different types of solar collectors and thermal cycles can be effectively utilized to overcome the drawbacks of traditional solar thermal power systems. This may provide a new feasible technology for lower cost, higher efficiency, large-scale solar thermal power generation.
  \item An air-water heat exchanger is applied in the cascade system to increase the temperature of the main steam temperature of the Rankine cycle. This provides a new way to overcome the shortcoming of the upper temperature limit of heat transfer oil in traditional solar trough systems, which helps to achieve higher Rankine cycle efficiency.
  \item Condensate of Rankine cycle is used to cool the Stirling engine. Rejected heat of the Stirling cycle can be reused by Rankine cycle, which helps to improve the overall system efficiency. 
  \item Multi-stage exergy lose reduction system is applied to reduce the temperature difference between oil and water in the steam generating system. The solar field can be divided into three independent sectors according to different states (vapor, vapor-liquid two phase, liquid) of water in the steam generating system. This also provides a new space for different types of solar collector technologies applied in different solar fields. For example, linear Fresnel reflectors or flat collectors can be applied for the preheating solar field to reduce costs; molten salt can be used as heat transfer fluid in the superheating solar field to increase the main steam temperature of the Rankine cycle.
  \item Influence of the arrangement of Stirling engine array in the cascade system is analyzed. In order to investigate the influence of connection types on SEA performance, five basic connection types of SEA were summarized according to the direction-irrelevant feature of Stirling engine. After
analyzing different factors on the performance of SEA, it is found that given heating and cooling fluids, using serial flow is the best choice for the connection type of an SEA.
\end{itemize}

\newpage
\section{Outlooks}
In this research, effective topologies of the proposed cascade system were designed, models of the systems developed based on the detailed component models, simulation of the cascade system and corresponding stand alone systems were carried out and the results were analyzed. However, there are many points valuable for further research.
\begin{itemize}
  %\item Solar power tower technology is gaining more and more attention, which represents the future of CSP technology and needs to be investigated in the future work. 
  \item Muti-stage exergy loss reduction system deserves more attention for its application of different kinds of collectors.
  \item Series connection of different collectors, such as flat plate and parabolic trough collectors, needs to be further studied to reduce the cost of solar power system.
  \item Economical analysis of the cascade system is required for the implement of the technology.
  \item Stirling engines are required for the platform to investigate the cascade utilization of solar energy.
\end{itemize}