\backmatter
\begin{ack}
%  This is the acknowledgment part.  
%  
%I have greatly appreciated the technical support, freedom, and detailed advice offered by my superviser Huang Shuhong, Inmaculada Arauzo and Zhang Yanping. They contributed much more time and effort to my research and reports than I expected. My technical skills have been greatly improved as a result of working and learning with them.
%  
%I have also been grateful for the flexibility offered by Nate Blair and Mark Mehos from the National Renewable Energy Laboratory (NREL) who provided the financial support for this research. I was able to research a large variety of topics related to Stirling dish systems with their continued support and interest.
%  
%Chuck Andraka from Sandia National Laboratories provided crucial data for validating the models in this research. His knowledge of Stirling dish systems and technical papers also allowed for a much better understanding of factors that affected the performance of these systems.
%  
%This project was financially supported by the National Renewable Energy Laboratory (NREL) under contract number ADC-5-55027-01.

%I offer the sincerest gratitude to my supervisors Prof. Wei Gao, Inmaculada Arauzo and Yanping Zhang. They contributed much more time and effort to my research and reports than I expected. My technical skills have been greatly improved as a result of working and learning with them.
%
%I also owe my gratitude to Prof. Shuhong Huang. It would not have been possible to complete my research on the solar thermal power without his help. Prof. Huang spent a lot of time for the coordination and technology improvement. 
%
%Thanks to my research partners, Chongzhe Zou, Xiaohong Huang, and Xiaolin Lei. They provided a good study atmosphere.
%Owing to their efforts, the cascade solar thermal test system is under construction and experiment investigation thereby  will be carried out.
%
%Thanks are given to the support of International S$\&$T Cooperation Program of China, under Grant No. 2014DFA60990. 
2004年,还是年少的我,作为本科生进入华中科技大学能源与动力工程学院,如今已过而立之年,博士论文修改工作的基本完成,预示着我在这里的学生生活即将结束。尽管这段难忘的人生时光曾经让我一度灰心丧气,但当我踉踉跄跄地要离开这片陪伴我成长的土地,我依旧感恩在这里经历的所有艰辛与快乐,因为这里有着我在学业和生活上的满满收获,并将改变我的人生行程。
对一个地方的不舍,源于那里有让我想念的人。感谢导师黄树红教授,师从恩师六年,在为人为学方面我受益良多。犹记当年保研初见与读博抉择之时,承蒙恩师不弃,硕博六年,跟随恩师进入太阳能光热发电研究领域。一路跌跌撞撞,如果学生在学术训练上有所长进的话,端赖恩师悉心指导。与您相处以来,深受您为学之风与为人之道的感染,您独到的学术见解与敏锐的学术思维,总让我豁然开朗,倍受启发。您离开后,思念恩师,每每论文遇阻,感到沮丧之时,莫名时常梦见恩师,想起您以往的教诲与鼓励,总能重新树立学术自信,静下心来投入论文撰写。师恩如山,静默无言,当面聆听恩师讲学论道是我珍藏的回忆,恩师深邃的学术思想与坚定的学术热爱将不断为我指引航向。
感谢在读期间于不同方面给予我指导和帮助的老师们。感谢张燕平副教授时常督促我的学习与写作,时刻关注论文的阶段性进展,并对论文行进中的困难给予及时的智慧点拨。感谢您在论文修改中,就篇幅结构安排、核心实验设计提出的诸多建设性意见。感谢高伟教授,从硕士阶段就     。感谢萨拉格萨大学的Professor Inmaculada Arauzo,有幸2015年跟随您到西班牙访学一年,开启全新的学术视野。您随和的为人、热情的问询,让我的访学生活温暖而充实。尤其在实验数据验证遇到困惑时,您耐心帮助我运用适切的计算方法,使我茅塞顿开。匆匆数年,论文撰写与实验操作几经曲折,学业的顺利完成,离不开各位老师的不吝教诲与鼓励。
《学记》有云,“独学而无友,则孤陋而寡闻”。在课题组这个学术团队中,我幸运地结识吕方明、王际洲。。。等同学,与他们共同交流学习,让我接触到许多闪光的学术观点。在于。。。。。。。等国际学友一起学习、一起生活的过程中,我不仅提高英语交流、写作能力,还收获了友谊。
最后,感谢家人的理解与支持。感谢我的父母含辛茹苦供我读书,一直做我的坚强后盾,只盼我学有所成。感谢我的妻子,在女儿桐桐出生后,挑起照料这个小生命的重担。感谢桐桐开心果,你的乖巧和可爱让我时刻感到贴心,充满动力。
而立之年,即将走出校园,这将是我人生道路的新起点。我将不忘初心、不负韶华,带着感恩与期许,用心经营好新的生活。
 
\end{ack}
