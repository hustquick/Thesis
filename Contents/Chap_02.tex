\chapter{System topology design}
% 利用热力学原理进行定性分析。提炼出创新点。
\section{Cascade collection}

Suzuki~\cite{Suzuki1986} analyzed the solar thermal systems with two different types of collectors connected in series. A key value of the collectors was revealed to be the key factor to determine whether a cascade system is better than either one of the collectors alone. The value is the product of the collector efficiency factor and the optical efficiency. If the value of the lower concentration ratio collector is lager than that of the higher concentration ratio, the cascade system is more effective. Furthermore, to obtain the maximum energy gain, there exists the optimum operating conditions.

Oshida and Suzuki~\cite{Oshida1987} presented the idea of optical cascade heat collection of solar energy. Two absorbers, one warm and the other hot, are used in the cascade system. The warm absorber is heated by the Fresnel lenses and the hot absorber is heated by CPC. HTF flows into the warm absorber firstly and then flows into the hot absorber. The temperature of HTF can increase more effectively. 
\nomenclature[C]{CPC}{Compound parabolic collector}

%Saltiel~\cite{Saltiel1988} studied the yearly performance of the multistage solar collector systems. Different types of collectors were used in the multistage solar collector systems. Flow control strategy was developed to simulate the performance of the systems under different climatic conditions. It was shown that the 
%
%A comparative study of the yearly performance of multistage solar collector systems, (comprised of more than one collector type) with a single on/off flow control strategy for all the collectors and separate on/off controls for each collector stage, is performed. Detailed numerical simulations under a range of climatic conditions showed that there is little advantage in using individual collector controls over a single on/off control strategy when the systems operate at low collector thresholds, but differences in system performance can be quite significant at high threshold values. In addition, the choice of the single control strategy (i.e., which collector the strategy is based on) at low thresholds is not critical in terms of system performance.

Kribus et al.~\cite{Kribus1999} proposed an idea of using separate aperture stages for different irradiance distribution.

A high-temperature solar thermal receiver is subject to temperature-dependent emission and convection losses. Minimizing these losses is essential to realization of high temperature, high efficiency systems. Dividing the aperture into separate stages according to the irradiance distribution has been shown theoretically to significantly reduce these losses. In such a partitioned system, the working fluid is gradually heated as it passes through a sequence of receiver elements with increasing irradiance levels. An experiment to demonstrate this principle using two heating stages has been constructed at the Weizmann Institute’s Solar Tower. The high-temperature receiver stage is the Directly Irradiated Annular Pressurized Receiver (DIAPR). The low-temperature stage is implemented as a partial ring of intermediate-temperature cavity tubular receivers (Preheaters) surrounding the central high-temperature stage. Following initial concentration by a part of the Weizmann Institute heliostat field, the light enters the receivers via secondary concentrators constructed as approximate CPCs. We present recent test results with the two-stage system. Air exit temperatures of up to 1000°C were obtained, with the low-temperature stage supplying up to 750°C. The power output was up to 55 kWth. Heat transfer in the high-temperature receiver, losses due to the partitioning, and future plans for partitioned receivers are discussed.

Collado2016~\cite{Collado2016}

In solar power tower (SPT) systems, selecting the optimum location of thousands of heliostats and the most profitable tower height and receiver size remains a challenge. Given the complexity of the problem, breaking the optimisation process down into two consecutive steps is suggested here; first, a primary, or energy, optimisation, which is practically independent of the cost models, and then a main, or economic, optimisation. The primary optimisation seeks a heliostat layout supplying the maximum annual incident energy for all the explored combinations of receiver sizes and tower heights. The annual electric output is then calculated as the combination of the incident energy and the simplified (annual averaged) receiver thermal losses and power efficiencies. Finally, the figure of merit of the main optimisation is the levelised cost of electric energy (LCOE) where the capital cost models used for the LCOE calculation are reported by the System Advisor Model (SAM)-NREL and Sandia. This structured optimisation, splitting energy procedures from economic ones, enables the organisation of a rather complex process, and it is not limited to any particular power tower code. Moreover, as the heliostat field layout is already fully optimised before the economic optimisation, the profiles of the LCOE versus the receiver radius for the tower heights explored here are sharp enough to establish optima easily. As an example of the new procedure, we present a full thermo-economic optimisation for the design of the collector field of an actual SPT system (Gemasolar, 20 MWe, radially staggered surrounding field with 2650 heliostats, 15 h of storage). The optimum design found for Gemasolar is reasonably consistent with the scarce open data. Finally, optimum designs are strongly dependent on the receiver cost, the electricity tariff and the assumed maximum receiver surface temperature.

Reddy~\cite{Reddy2009}

Numerical analysis of solar dish modified cavity receiver with Cone, CPC and Trumpet reflectors is presented. Three-dimensional modeling is carried out to estimate the convective and radiative heat loss from the receiver for different angles of inclination and operating temperatures. Incorporating reflectors in the modified cavity receiver for second stage concentration, the natural convection heat losses are reduced by 29.23, 19.81 and 19.16\%, respectively. The receiver with the trumpet reflector has shown better performance as compared to other configurations.

Mills~\cite{Mills1995}

Maximally concentrating collectors include the class of ideal concentrating collectors, but are a more general class offering many more practical possibilities. By evaluating such configurations using the concept of maximal flux concentration, based upon average radiation flux concentration over the acceptance angle, clear ray trace comparisons may be made between different collector configurations. These comparisons allow the most effective configuration to be selected for a given application. An example of a comparatively simple and practical two-stage concentrator having equal or better maximal performance than previous work for high rim angle primaries is given. This uses an unusual straight section of reflector and allows rays to cross from one reflector segment of the secondary to another. Versions which allow concentration up to 90\% of maximal are described, as are versions achieving 80\% with high collection efficiency. Use of the ~~~ geometrical concentration criterion based upon aperture ratios is suggested to be inappropriate for comparisons.

Collado~\cite{Collado2016}

Abstract In solar power tower (SPT) systems, selecting the optimum location of thousands of heliostats and the most profitable tower height and receiver size remains a challenge. Given the complexity of the problem, breaking the optimisation process down into two consecutive steps is suggested here; first, a primary, or energy, optimisation, which is practically independent of the cost models, and then a main, or economic, optimisation. The primary optimisation seeks a heliostat layout supplying the maximum annual incident energy for all the explored combinations of receiver sizes and tower heights. The annual electric output is then calculated as the combination of the incident energy and the simplified (annual averaged) receiver thermal losses and power efficiencies. Finally, the figure of merit of the main optimisation is the levelised cost of electric energy (LCOE) where the capital cost models used for the \{LCOE\} calculation are reported by the System Advisor Model (SAM)-NREL and Sandia. This structured optimisation, splitting energy procedures from economic ones, enables the organisation of a rather complex process, and it is not limited to any particular power tower code. Moreover, as the heliostat field layout is already fully optimised before the economic optimisation, the profiles of the \{LCOE\} versus the receiver radius for the tower heights explored here are sharp enough to establish optima easily. As an example of the new procedure, we present a full thermo-economic optimisation for the design of the collector field of an actual \{SPT\} system (Gemasolar, 20 MWe, radially staggered surrounding field with 2650 heliostats, 15 h of storage). The optimum design found for Gemasolar is reasonably consistent with the scarce open data. Finally, optimum designs are strongly dependent on the receiver cost, the electricity tariff and the assumed maximum receiver surface temperature. 

Gordon and Saltiel~\cite{Gordon1986}

We present an analytic method for predicting the long-term performance of solar energy systems with more than one collector brand (“multi-stage” systems). This procedure enables the designer to determine the most cost-effective method of combining different collector brands for a given load. Although our derivations pertain to solar systems for constant load applications and/or near constant collector operating threshold, they can also be used for conventional multi-pass designs. The problems of excess energy delivery, and of various collector on/off control strategies, are taken into account. Our results are simple closed-form expressions whose evaluation requires readily-available average climatic data, and load and collector characteristics. The analytic method is illustrated by a solved example which shows that significant savings can be realized by combining different collector brands for a given application (multi-staging).

空气槽式集热器

Good et al.~\cite{Good2015}

An entirely novel solar receiver design for solar trough concentrators is proposed using air as heat transfer fluid at operating temperatures exceeding 600 °C. It consists of an array of helically coiled absorber tubes contained side-by-side within an insulated groove having a rectangular windowed opening. Secondary concentrating optics are incorporated to boost the geometric concentration ratio to 97×. The multiple absorber tubes are connected via two axial pipes serving as feeding and collecting manifolds. The steady-state energy conservation equation coupling radiation, convection, and conduction is formulated and solved numerically using the finite volume technique. The solar flux distribution incident at each absorber tube is determined by Monte Carlo ray-tracing using spectrally and directionally dependent optical properties. Thermal radiative heat exchange is analyzed using the gray-band approximated radiosity method for an enclosure with a selective window. Model validation is accomplished by comparison to on-sun experiments with a 1 m-long solar receiver prototype composed of 7 absorber tubes, mounted on a 4.85 m-aperture solar trough concentrator. Feeding rates in the range of 5–20 ln/min to each absorber tube led to air outlet temperatures of 621–449 °C and a peak receiver efficiency of 64\%.

Barder et al.~\cite{Bader2015}

A tubular cavity-receiver that uses air as the heat transfer fluid is evaluated numerically using a validated heat transfer model. The receiver is designed for use on a large-span (9 m net concentrator aperture width) solar parabolic trough concentrator. Through the combination of a parabolic primary concentrator with a nonimaging secondary concentrator, the collector reaches a solar concentration ratio of 97.5. Four different receiver configurations are considered, with smooth or V-corrugated absorber tube and single- or double-glazed aperture window. The collector's performance is characterized by its optical efficiency and heat loss. The optical efficiency is determined with the Monte Carlo ray-tracing method. Radiative heat exchange inside the receiver is calculated with the net radiation method. The 2D steady-state energy equation, which couples conductive, convective, and radiative heat transfer, is solved for the solid domains of the receiver cross-section, using finite-volume techniques. Simulations for Sevilla/Spain at the summer solstice at solar noon (direct normal solar irradiance: 847 W$\cdot$m$^{-2}$, solar incidence angle: 13.9$^\circ$) yield collector efficiencies between 60\% and 65\% at a heat transfer fluid temperature of 125$^\circ{}C$ and between 37\% and 42\% at 500$^\circ{}C$, depending on the receiver configuration. The optical losses amount to more than 30\% of the incident solar radiation and constitute the largest source of energy loss. For a 200 m long collector module operated between 300 and 500$^\circ{}C$, the isentropic pumping power required to pump the \{HTF\} through the receiver is between 11 and 17 kW. 

多种集热器的混合利用

Some researchers have investigated the combination of different types of collectors for CSP.
%Cau~\cite{Cau2014} reported a comparative performance analysis of CSP plants using both parabolic trough and linear Fresnel collectors. A two-tank direct thermal storage system are included and in the Rankine cycle, regenerator, $4\sim6$ steam extractions and air-cooler condenser are used.
Desai et al.~\cite{Desai2015} presented an integrated CSP plant configuration with the combination of both PTC and LFC. Thermo-economic comparisons between PTC-based, LFC-based and integrated CSP plant configurations, without hybridization and storage, were analyzed.  It is demonstrated that the cost of energy of an integrated CSP plant is $9.6\,\%$ cheaper than PTC-based CSP plant and $13.5\,\%$ cheaper than LFR-based CSP plant.
Coco et al.~\cite{Coco2015} developed four different line-focusing solar power plant configurations integrated both direct steam generation and Brayton power cycle. In these configurations, collectors are divided into different solar fields to supply different heat demands. This provides the ability to use different types of collectors (parabolic trough and linear Fresnel) in the systems.
\nomenclature[C]{SRC}{Steam Rankine Cycle}
\nomenclature[C]{ISCC}{Integrated Solar Combined Cycle} 

\section{Cascade utilization}

利用槽式集热器为塔式电站的给水预热
朗肯循环、斯特林循环综合利用
多级朗肯循环

Many researchers have done the work on the combination of different thermodynamic cycles for CSP. Lots of the work focused on integrated solar combined cycle (ISCC) with parabolic trough, where Rankine cycle is used as the bottom cycle. 
Li and Yang~\cite{Li2014} proposed a novel two-stage ISCC system that could reach up to 30\% of the net solar-to-electricity efficiency. In their research, the impact on the system overall efficiencies of how and where solar energy is input into ISCC system was investigated.
Behar et al.~\cite{Behar2014} reviewed the R\&D activities and published studies since the introduction of such a concept in the 1990s. One of the conclusions is that the higher the solar radiation intensity the better is the performance of the ISCCS than those of conventional CSP technologies.
Gulen ~\cite{Gulen2015} used the exergy concept of the second law of thermodynamics to distill the complex optimization of ISCCS to its bare essentials. After the exergy analysis, physics-based, user-friendly guidelines were provided to help direct studies involving heavy use of time consuming system models in a focused manner and evaluate the results critically to arrive at feasible ISCC designs.
Shaaban ~\cite{Shaaban2016} introduced a novel ISCC with steam and organic Rankine cycles. The ORC was used in order to intercool the compressed air and produce a net power from the received thermal energy. The proposed cycle performance was studied and optimised with different ORC working fluids.
Alqahtani and Dalia~\cite{Alqahtani2016} quantified the economic and environmental benefits of an ISCC power plant relative to a stand-alone CSP with energy storage, and a natural gas-fired combined cycle plant. Results show that integrating the CSP into an ISCC reduces the LCOE of solar-generated electricity by 35-40\% relative to a stand-alone CSP plant, and provides the additional benefit of dispatch ability.
Manente~\cite{Manente2016} developed a $390\,\mathrm{MWe}$ three pressure level natural gas combined cycle to evaluate different integration schemes of ISCC. Both power boosting and fuel saving operation strategies were analyzed in the search for the highest annual efficiency and solar share. Result shown that, compared to power boosting, the fuel saving strategy shows lower thermal efficiencies of the integrated solar combined cycle due to the efficiency drop of gas turbine at reduced loads.
Rovira et al.~\cite{Rovira2016} compared the annual performance and economic feasibility of ISCC using two solar concentrating technologies: parabolic trough collectors (PTC) and linear Fresnel collectors (LFC). Different configurations were considered and results shown that only evaporative configuration is the most suitable choice.
\nomenclature[C]{PTC}{Parabolic Trough Collector}
\nomenclature[C]{LFC}{Linear Fresnel Collector}
Compared with traditional ISCC design, two new conceptual hybrid designs for ISCC with parabolic trough were represented by Turchi et al.~\cite{Turchi2014}. In the first design, gas turbine waste heat is supplied for both heat transfer fluid heating and feed water preheating. In the second design, gas turbine waste heat is supplied for a thermal energy storage system.
Mukhopadhyay and Ghosh~\cite{Mukhopadhyay2016} presented a conceptual configuration of a solar power tower combined heat and power plant with a topping air Brayton cycle. The conventional gas turbine combustion chamber is replaced with a solar receiver. A simple downstream Rankine cycle with a heat recovery steam generator and a process heater have been considered for integration with the solar Brayton cycle.
Li et al.~\cite{Li2016a} presented a novel cascade system using both steam Rankine cycle (SRC) and organic Rankine cycle (ORC). Screw expander is employed in the steam Rankine cycle for its good applicability in power conversion with steam-liquid mixture. The heat released by steam condensation is used to drive the ORC.
\nomenclature[C]{ORC}{Organic Rankine Cycle}
Al-Sulaiman~\cite{AlSulaiman2014} compared the produced power of an SRC-ORC combined cycle with traditional SRC cycle. The SRC is driven by parabolic trough solar collectors, and the ORC cycle is driven by the condensation heat of the SRC.
Bahari et al.~\cite{Bahari2016} considered the optimization of an integrated system using organic Rankine cycle to utilize the heat released by the Stirling cycle. However, the integrated system is a primitive design and it didn't consider the application in CSP field.

\section{System topology selection}
各种拓扑结构分析