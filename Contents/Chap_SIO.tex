\chapter{总结与展望}
\section{总结}

考虑到现有太阳能光热发电技术的优缺点,本文提出,组建并优化太阳能梯级集热发电系统,为大规模低成本高效率利用太阳能光热发电技术提供新的可行方案。本文的研究内容和相关贡献主要有如下几点:

\begin{enumerate}[label=(\arabic*)]
  \item 提出了有效的太阳能梯级集热发电的拓扑结构。通过分析太阳能光热发电系统中各部件的热力学特征及其运行参数,本文系统性地分析了太阳能梯级集热发电系统设计中的各种技术方案,并针对这些方案逐个进行了深入的研究。经过分析和排除,确定了两种有效的太阳能梯级集热发电的拓扑结构。这两种拓扑结构中,同时采用了多种集热器和多个热力循环,以实现能量的梯级收集和梯级利用。在一种拓扑结构中,同时选用朗肯循环和斯特林循环用于发电,利用朗肯循环的凝结液来冷却斯特林机,以回收利用斯特林循环放出的热量。在另一种拓扑结构中,采用多级有机工质朗肯循环的方式,顶部循环的凝结热被底部循环吸收用于发电。
  \item 建立了太阳能光热发电系统中各部件的机理模型。本文详细介绍了太阳能光热发电系统中关键部件的建模研究,包括:槽式集热器、碟式集热器、斯特林机、蒸汽发生系统、朗肯循环子系统等的机理建模研究。各部件的机理模型是由经典理论所证实的或是经过合理简化假设并被大量实验数据所验证的。
  
针对槽式集热器,进行了整体换热系数沿焦线方向均匀一致的假设,经过对集热器的热力学分析,得到了计算集热器效率的简化方程,建立了槽式集热器的机理模型,并利用实验所得的数据对模型进行了验证。
  
  针对碟式集热器,建立了碟式接收器的热网络模型,通过对模型各项热损失的分析计算,建立了碟式集热器的机理模型,并利用实验所得的数据对模型进行了验证。
  
  针对斯特林机,基于合理的简化和假设,建立了考虑多种不可逆因素和热损失的模型,并经过对GPU-3型斯特林机的模拟,同经典斯特林机模型和实验数据进行了对比分析。
  \item 开发了具有自主计算机知识产权的太阳能光热发电系统设计软件,并创建了太阳能光热梯级集热发电系统的模型。太阳能光热发电系统建模设计软件采用面向对象的方式,利用自底向上的设计方法,按照设备-子系统-全系统的顺序,利用部件的机理模型,组建太阳能光热发电系统。所建立的模型具有易于搭建,结构清晰,便于替换或改进部件等优点。
  \item 针对斯特林机组的排列方式进行了优化工作。斯特林机组的排列方式对梯级系统的性能有很大的影响,为了研究并优化斯特林机组的排列方式,本文依据斯特林机的工作特性,提出了五种基本的斯特林机组排列方式,并对这五种排列方式进行了建模仿真分析。通过分析不同运行参数的影响,得出了串联连接的斯特林机组具有最佳性能和最佳健壮性的结论。
  \item 针对蒸汽发生系统进行了优化工作。针对传统槽式太阳能电厂的蒸汽发生系统存在较大㶲损的缺点,提出了新的分段加热系统。并通过系统建模仿真,对传统蒸汽发生系统和分段加热系统进行了对比分析。结果表明,分段加热系统可以有效降低蒸汽发生过程中的传热温差,进而减少传热过程中产生的㶲损,同时提高太阳能场的集热效率。与传统的蒸汽发生系统相比,本文给出的三种不同分段加热系统方案可使蒸汽发生过程的㶲损失减少14.3\%到76.7\%,太阳能场的集热效率也会提高0.9\%到3.6\%。
  \item 针对太阳能梯级集热发电系统建立了对应的独立系统,并对二者进行了多种参数影响下的对比分析。本文提出了梯级系统的性能评估方法和对应独立系统的选取方案,建立了梯级系统及其对应独立系统的模型,进行了多种参数条件下的模拟分析。结果表明,在太阳法向直射强度大于550\,$\mathrm{W/m^2}$及碟式集热器出口空气温度为1073$\,\mathrm{K}$时,所提出的梯级系统比其对应的独立系统的发电效率高。
	\item 搭建了太阳能聚光集热实验台,并进行了相关的实验工作。针对太阳辐射的不可控性,连续性及多变性,设计了特殊的实验工况,并分析了太阳法向直射强度、传热流体流量、入口温度对集热器效率的影响,并利用实验结果对同本文提出的模型进行了验证分析。结果表明,本文所建立的集热器模型具有良好的性能趋势预测能力,预测结果具有较小的相对误差。
\end{enumerate}

\section{创新点}

本文主要有以下创新点:
\begin{enumerate}[label=(\arabic*)]
  \item 提出了同时采用多种型式的集热器和多种形式的热力循环的梯级系统。\replaced[id=ZC]{利用不同型式的集热器的集热特性实现梯级加热,利用不同的热力循环实现能量的梯级利用。不同回路间还可以通过热量交换来实现能量的梯级利用。梯级系统可以克服传统太阳能光热发电系统中的缺点,为大规模低成本高效率太阳能光热发电提供新方向。}{这样,不同工作特性的集热器和不同工作温度区间的热力循环可以有效地组合起来,进而克服传统太阳能光热发电系统中的缺点。这可能为大规模低成本高效率太阳能光热发电提供新的方向。}

  \deleted[id=ZC]{(2) 提出的梯级系统中,采用了空气-水换热器来提高朗肯循环的主汽温度。这提供了一种新的方式来克服传统的以导热油为传热流体的槽式系统主汽温度受限于导热油极限温度的缺陷,这为朗肯循环的发电效率提供了提升空间。}
  \item \added[id=ZC]{提出了一种梯级系统}\deleted[id=ZC]{提出的梯级系统中},采用朗肯循环的凝结液来冷却斯特林机,\deleted[id=ZC]{进而}有效利用斯特林机放出的热量\added[id=ZC]{,增加朗肯循环的能量输入来源}。通过回收这部分能量,可以有效提升梯级系统的整体发电效率。
  \item 提出了采用分段加热系统来\replaced[id=ZC]{减少}{降低}蒸汽发生过程中的\replaced[id=ZC]{㶲损}{温差}的方案。\replaced[id=ZC]{将太阳能场划分为三个片区,各片区分别独立为预热器、蒸发器和过热器供热。通过调节各片区的传热流体的流量可以有效降低蒸汽发生过程的换热温差,减少蒸汽发生过程的㶲损。}{通过将太阳能场划分为三个片区的方式,分别调节预热区、蒸发区和过热区所对应的导热油的质量流量,进而有效降低传热过程的温差。同时,}这也为在太阳能场中使用多种集热技术提供了空间。
\end{enumerate}

\section{展望}

本文所做的研究只是在寻找大规模低成本高效率的光热发电技术方案的研究上做了一些初步的探索,还有很多问题尚待未来的研究工作继续展开:
\begin{enumerate}[label=(\arabic*)]
  \item 多种集热方式集成方案需要进一步深入研究。尤其是\autoref{sec:csc}提到的应用于塔式太阳能技术的利用槽式集热器和平板式加热器进行预热的方案。该方案可以有效利用不同形式的集热器的工作特性,有效降低光热发电系统的单位发电成本。
  \item 分段加热系统可以进行更加深入的研究。研究不同太阳能场片区采用不同种类的集热器给系统带来的收益。预热器对应的片区可以采用平板式集热器或菲涅尔反射镜来降低低温区的集热成本。蒸发器对应的片区可以选用大量槽式集热器并联连接以提高导热油的流量,同时,该片区对温升的要求很低,所以不需要串联集热器。过热器对应的片区可以选用集热温度可以达到很高的熔融盐作为太阳能槽式集热器的传热介质以提升朗肯循环的主汽温度。
  \item \added[id=ZC]{对梯级电厂的全年性能进行分析。考虑到太阳能辐射资源的周期变化性,对太阳能热电厂进行全年性能评估是非常重要的。}
  \item \deleted[id=ZC]{需要}对梯级系统进行经济性分析。一种新的技术方案在应用之前一定要进行经济性分析,经济上可行是新方案大规模应用的必要条件。
  \item 太阳能光热发电平台仍需进一步完善,为太阳能梯级示范系统的搭建做好准备工作。有机工质朗肯循环系统的汽轮机轴承需要及时更换,实验平台的操作界面及数据采集功能还有很大的优化空间。
\end{enumerate}
\nomenclature[S]{$o$}{出口}
\nomenclature[S]{$i$}{进口}