\zhabstract{

随着化石能源消耗和环境污染问题的凸显,太阳能被广泛认为是未来最有潜力替代传统化石能源的清洁能源。本文以国家国际合作项目专项“太阳能梯级集热发电系统关键技术合作研究”为背景,目标是研究太阳能光热发电装置,利用各种传统型式的太阳能光热发电系统的优缺点以及热力特性,提出并组建,优化太阳能梯级集热发电系统,为探索大规模低成本高效率利用太阳能的光热发电技术提供新的方案。主要研究内容包括:

提出了多种新颖的采用梯级集热和梯级发电的新型太阳能梯级集热发电系统。在梯级系统中,采用多种型式的集热器,实现能量的梯级收集,采用多种形式的热力循环,实现能量的梯级利用。经过对梯级系统的各技术方案逐个研究分析,确定了具有代表性的太阳能梯级集热发电系统方案。

采用数学计算工具和系统开发工具,建立了梯级系统中各部件的机理模型,进而开发了太阳能光热发电系统设计软件。采用面向对象的方法,保证了各部件之间既具有独立性又具有关联性,系统模型具有方便搭建,结构清晰,易于改进等优点。其中,斯特林机的建模过程中,考虑了多种不可逆过程及多类损失,建立了较为完善的斯特林机机理模型,并进行了模型验证分析。结果表明,所建立的斯特林机模型的精度要高于传统的经典斯特林机模型。

研究了太阳能梯级集热发电系统中斯特林机组不同排列方式对系统效率的影响。针对斯特林机的工作特点,提出了五种基本的斯特林机组排列方式,并为这五种排列方式建立了仿真模型,分析了不同流体入口温度、热容量及斯特林机数目的条件下,各种排列方式的性能差异。发现串联连接是最佳的排列方式,斯特林机组具有最佳健壮性和最大的发电效率。

提出了分段加热系统,有效降低了蒸汽发生系统中的㶲损。在传统蒸汽发生系统的换热过程中,加热流体无相变,被加热流体有相变,两者存在较大的换热温差,换热过程有较大的㶲损。本文提出分段加热的方法,通过改变各段加热流体的流量,减小换热温差,降低换热过程的㶲损,并能有效提升太阳能镜场的集热效率。

提出了太阳能梯级集热发电系统性能评估方法。本文针对新型梯级发电系统提出了与传统型式太阳能光热发电独立系统的对比方案,并建立了系统性能计算模型。经过建模仿真分析发现,梯级系统在一定的参数条件下,相比其对应的独立系统,具有更高的总体光电转换效率。

建立了太阳能光热发电实验平台,并开展了相关的实验工作。通过设计不同的实验工况,探究了太阳法向直射强度,传热流体流量,入口温度对集热器热性能的影响。实验还验证了本文建立的槽式集热器和碟式集热器模型。

本文提出了太阳能梯级集热和梯级利用的概念,研究了光热系统的建模设计方法,分析了梯级系统斯特林机组排列方式和蒸汽发生系统的优化,并对梯级系统进行了性能评估,参与了太阳能发电试验台的建设和试验研究工作。
}
\zhkeywords{槽式集热器,碟式集热器,朗肯循环,斯特林循环,斯特林机组,梯级光热发电}

\enabstract
{
With the increasing awareness of the problem of fossil energy consumption and environmental pollution, solar energy is regarded as the best potential alternative of fossil energy. This research is based on the national cooperation project ``Collaborative research on key technologies to produce electricity by cascade utilization solar thermal energy''. The objective of this project is to conduct research on the equipment of solar thermal power generation system, to propose, develop and optimize a solar thermal cascade system depending on the advantages and disadvantages of the solar thermal power generation technologies, and to explore a new feasible technology for large-scale solar thermal power generation. The main contents and conclusions of this thesis are as follows:

Multiple novel topological structures with cascade collection and cascade utilization of the cascade systems were proposed. In these systems, different types of collectors were used for cascade collection and different types of thermodynamic cycles were used for cascade utilization. After the investigation of each technical proposal of cascade system, representative typical cascade solar thermal power system topologies were selected.

Mechanism models were established for the components of solar thermal power generation system by using mathematical calculation tool and system development tool. The modeling process uses an object-oriented approach to ensure each component has both independence and relevance. The system model has the advantages of convenient organization, clear structure, easy improvement. For component modeling, the Stirling machine modeling process, considering various irreversibilities and losses, established a more accurate Stirling mechanism model with verification analysis. The results show that the accuracy of the established Stirling model is higher than that of the classical classical Stirling engine models.

The effect of different arrangements of Stirling engines on the efficiency of the cascade system was studied. 
According to the working features of Stirling engine, five basic arrangements of Stirling engine array were proposed, and corresponding simulation models were established.
Performance differences of different arrangements of Stirling engine array were analyzed with different inlet fluid temperature, fluid heat capacity and Stirling engine numbers.
It was found that series connection is the best arrangement for the best robustness and maximum efficiency of the Stirling engines.

A multistage heating system was proposed, which can effectively reduce the exergy loss of steam generating system. During the entire heat exchange process of a conventional steam generating system, there is no phase change in the heating fluid and there is a phase change in the heated fluid. There exist large heat transfer temperature differences between the two fluids in the heat exchangers, which makes large entropy production during the heat exchange process. In this thesis, a method of heating in stages is proposed, in which the flow rates of the heating fluid in different heat exchangers are controlled to reduce the heat transfer temperature difference and the exergy losses, and effectively increase thermal efficiency of solar fields.

A performance evaluation method of solar cascade thermal power generation system was proposed. In this thesis, corresponding independent systems of the cascade system were chosen for comparison, and the system performance evaluation models were established. After simulation and result analysis of the systems, it can be found that the cascade system has a higher overall solar-to-electric conversion efficiency under certain parameters compared to its corresponding independent systems.

A solar thermal power generation test platform was established, and the relevant experimental work was carried out. By designing different experimental conditions, influences of solar direct normal irradiance, flow rate and inlet temperature of heat transfer fluid were investigated.
The experiment also validated the established trough collector and dish collector models.

In this thesis, the idea of solar thermal cascade collection and cascade utilization was put forward. The mechanism research and modeling method of solar thermal system were studied. The Stirling engine array arrangement and steam generating system were optimized. The performance evaluation of cascade system was investigated. Construction work and experimental research of the solar thermal power generation test platform were carried out.

%With the increasing awareness of the problem of fossil energy consumption and environmental pollution, solar energy as a renewable energy, which has the advantages of widely distribution, huge amount, inexhaustible and no pollution, has received much attention by many countries and been regarded as the best potential candidate of fossil energy. Concentrating solar thermal power generation is another form of solar power generation technology except solar photovoltaic power generation. Compared to solar photovoltaic, solar thermal power is gaining more attention for its advantages as smooth power generation, good grid compatibility, easy to integrate with existing fossil power plant. However, solar thermal power is not yet widely applied due to the problems of current technologies.
%
%For this reason, this research is based on the national cooperation project "Collaborative research on key technologies to produce electricity by cascade utilization solar thermal energy" as the background. The objective of this project is to research the equipment of solar thermal power generation system, to propose, develop and optimize a solar thermal cascade system depending on the advantages and disadvantages of the solar thermal power generation technologies, and to explore a new feasible technology for large-scale solar thermal power generation. The main contents and conclusions of this thesis are as follows:

%Multiple topological structures with cascade collection and cascade utilization of the cascade systems were proposed. In these systems, different types of collectors were used for cascade collection and different types of thermodynamic cycles were used for cascade utilization. After system evaluation, parameter selection, preliminary calculation and scheme comparison, two representative typical schemes were determined. In one scheme, both Rankine cycle (water as the working fluid) and Stirling cycle are used for power generation. Cooling water of the Rankine cycle is used to cool the hot end of the Stirling engines to recover the released heat. In the other scheme, multiple organic Rankine cycles are used for power generation. Condensation heat of upper cycle is absorbed by lower cycle for energy cascade utilization.

%Mechanism models were established for the components of solar thermal power generation system by using EES and MATLAB. The modeling process uses an object-oriented approach, taking full advantage of inheritance, polymorphism and other characteristics, to ensure that each component has both independence and relevance. Among them, the Stirling machine modeling process, considering various irreversibilities and losses, established a more accurate Stirling mechanism model with verification analysis. The results show that the accuracy of the established Stirling model is higher than that of the classical classical Stirling engine models.

%The effect of different arrangements of Stirling engines on the efficiency of the cascade system was studied. Stirling engines can be connected in series, in parallel or in hybrid. The connection type affects the temperature and flow of the heating and cooling fluids of each engine, which in turn affects the power and efficiency of the Stirling machines, and the solar-to-electric efficiency of the cascade system. Through the analysis of different arrangements of Stirling engines, it was found that series connection is the best connection type for the best robustness and maximum efficiency of the Stirling engines, and the largest solar-to-electric efficiency of the cascade system.

%A method of multistage heating was proposed, which can effectively reduce the exergy loss of steam generating system. In conventional steam generation systems, the flows of heating fluid (typically oil) and heated fluid (typically water) through the preheater, evaporator and superheater do not change. In the entire heat exchange process, there is no phase change in the heating fluid, while there is a phase change in the heated fluid. There exist large heat transfer temperature differences between the two fluids in the heat exchangers, which makes large entropy production during the heat exchange process. In this paper, a method of heating in stages is proposed, in which the flow rates of the heating fluid in different heat exchangers are controlled to reduce the heat transfer temperature difference and the exergy losses. 

%A comparison method of cascade system and traditional solar thermal power generation systems is proposed. In this paper, corresponding independent systems of the cascade system was proposed for comparison. It is found that the cascade system has a higher overall solar-to-electric conversion efficiency under certain parameters compared to its corresponding independent systems. Under the condition of direct solar radiation intensity of 700\,$\mathrm{W/m^2}$ and dish collector outlet air temperature of 800$\mathrm{^\circ C}$, the solar thermal cascade power generation system of Scheme 1 is better than the corresponding The efficiency of stand-alone system is increased by 5.2\%. The solar thermal cascade power generation system selected in Scheme 2 is 15.3\% more efficient than the corresponding independent system.

}
\enkeywords
{parabolic trough collector, parabolic dish collector, Rankine cycle, Stirling cycle, Stirling engine array, cascade solar thermal power}